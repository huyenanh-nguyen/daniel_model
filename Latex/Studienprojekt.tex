\documentclass[oneside,10pt,a4paper]{report}
\usepackage[a4paper, left=3cm, right=3cm, top=3cm, bottom=3cm, headsep=10mm, footskip=12mm]{geometry}
\usepackage[T1]{fontenc}
\usepackage[ngerman, english]{babel}    % mehrsprachiger Textsatz
% babel: letzte Sprache in Optionen zeigt die Sprache des Dokumentes
% und kann durch den Befehl \selectlanguage{} geaendert werden
% Passen Sie die Optionen des babel-Paketes nach Bedarf an!
\usepackage{float}
\usepackage{graphicx}
\usepackage{url}
\usepackage{pdflscape}
\usepackage{mathtools}
\usepackage{amssymb, amsmath, amstext}
\usepackage{amsthm}
\usepackage{xcolor}
\usepackage{nameref}
\usepackage{siunitx}
\usepackage{makecell}
\usepackage{hyperref}
\usepackage{enumitem}
\usepackage[superscript,biblabel]{cite}
\usepackage{caption}
\usepackage{subcaption}
\usepackage{tabularx} 			% Tabellen erzeugen
\usepackage{multirow}			 % Zeilen in Tabellenbearbeitung
\usepackage{multicol} 			% Spalten in Tabellenbearbeitung 
\usepackage{lmodern}                        % Ersatz fuer Computer Modern-Schriften 
\usepackage{amsmath}                                           % zum besseren Aussehen am Bildschirm
\usepackage{booktabs} % für schönere Tabellen
\usepackage{sidecap}
\usepackage{rotating} % für die Landscape-Umgebung
\usepackage{afterpage}
\definecolor{Bluetitle}{HTML}{1F3864}
\definecolor{softbluetitle}{HTML}{274D7E}
\definecolor{Greyish}{HTML}{5A5A5A}
%\renewcommand{\refname}{Reference}
\usepackage{array,multirow}
\newcommand{\specialcell}[2][c]{%
	\begin{tabular}[#1]{@{}c@{}}#2\end{tabular}}
\usepackage{titlesec}

\titleformat{\chapter}[display]
{\normalfont\bfseries}{}{0pt}{\Huge}

\usepackage{lipsum} 


\begin{document}
	
	\begin{titlepage}
		\begin{center}
			\begin{figure}[h!tbp]
				\includegraphics[width=\linewidth]{HUlogo.PNG}
			\end{figure}
			\vspace*{2 cm}
			
			\textcolor{Bluetitle}{\textbf{\huge Molekulare Mikrobiologie}}\par
			
			\vspace*{2cm}
			\textcolor{Greyish}{\textbf{Versuchsdurchführende}}\par
			\textcolor{Greyish}{Huyen Anh Nguyen (572309)}\par
			\textcolor{Greyish}{Johanna Laetitia Heide (...)}\par
			
			
			\vspace*{0.5cm}
			\textcolor{Greyish}{\textbf{Versuchsort}}\par
			\textcolor{Greyish}{Haus 14, Kursraum}\par
			\textcolor{Greyish}{Gruppe 3}\par
			
			
			\vspace*{2 cm}
			\textcolor{Greyish}{\textbf{Versuchsleiter}}\par
			\textcolor{Greyish}{Prof. Dr. Marc Erhardt}\par
			\vspace*{0.5cm}
			\textcolor{Greyish}{\textbf{Versuchsbetreuer}}\par
			\textcolor{Greyish}{Dr. Caroline Kühne}\par
			\textcolor{Greyish}{Heidi Landmesser}\par
			
			
			\vspace*{2 cm}
			\textcolor{Greyish}{Abgabe 31. Januar 2025}\par
			
			
			
		\end{center}
	\end{titlepage}
	
	
	\tableofcontents
	\chapter{Note from the Author}
	Wir erkläres ausdrücklich, dass wir die Anmerkungen zur Anfertigung des Protokolls gelesen und befolgt haben, dass es sich bei der von uns eingereichte Arbeit um eine von uns erstmalig, selbstständig ohne fremde Hilfe verfasste Arbeit handelt und dass wir sämtliche verwendete zulässige Literatur (Fachpublikationen/-bücher), die unverändert oder abgewandelt wiedergeben werden, insbesondere Quellen für Texte, Grafiken, Tabellen und Bilder als solche kenntlich gemacht haben.\\
	Uns ist bewusst, dass Verstöße gegen diese Grundsätze als Täuschung betrachtet und entsprechend der Prüfungsordnung und/oder der Fächerübergreifenden Satzung zur Reglung von Zulassung, Studium und Prüfung der Humboldt-Universität zu Berlin geahndet werden.
	\\
	\\
	Die Versuche werden wie im Skript "Praktikum Molekulare Mikrobiologie" von Dr. Caroline Kühne \cite{Mibi-Script} durchgeführt. Nur bei Abweichungen werden diese in diesen Protokoll aufgeführt. Zudem wird in diesen Protokoll für die einzelnen Versuchen, die verwendete Phagen- und Sallmonella-Stämme aufgelistet, da in den meisten Versuchen, die unterschiedlichen Stämme unter den Studentenen-Gruppen aufgeteilt wurden.
	
	
	
	
	
	\addcontentsline{toc}{section}{Bibliography}
	\bibliographystyle{plainurl}
	\nocite{*}
	\bibliography{Literatur}
	\newpage
\end{document}